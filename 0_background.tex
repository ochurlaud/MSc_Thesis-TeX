\chapter{Background}
\label{sec:background}

\section{BESSY II}

\subsection{General presentation}
BESSY II (\textbf{B}erliner \textbf{E}lektronen\-\textbf{S}peicherring-Gesellschaft für \textbf{Sy}n\-chro\-tron\-strahlung m.b.H.) is Berlin electron storage ring, aimed at producing high energy light ray by synchrotron radiation. It emits extremely brillant photons pulse ranging from the long wave terahertz region to hard X rays, with an emphasis on the soft X-ray range~\cite{web:bessy_homepage}. Scientific projects can freely apply to an experimental station, where they are able to adjust the wavelength, polarization and photon energy. More than 2000 scientists per year are using BESSY II equipment.

The storage ring has a circumference of 240~meters and provides around 50 beamlines (paths of light rays between the accelerator and experimental stations). The electrons are accelerated to an energy up to 1.7 GeV.

BESSY II was inaugurated in 1998 and is since 2009 a facility of the \textit{Helmholtz-Zentrum Berlin für Materialien und Energie} (HZB), to study material structures and processes by guest scientists.

Additionally to the guest scientists, \todo{I don't know how to call them}experts and operators ensure the good functioning of the whole facility and work on refining the quality and the stability of the light rays. 

\subsection{General functioning}
The way BESSY II functions is based on the synchrotron radiation phenomenon: any accelerated particle emits radiations (in the form of photons), with a maximal amplitude in the case of a circular acceleration. \todo{Should I detail the equations?} It can be shown\cite{book:wille} that the radiated power in circular acceleration can be given as 
\begin{equation}
P_s = \frac{e^2 c}{6 \pi \epsilon_0}\frac{1}{(m_0 c^2)^4}\frac{E^4}{R^2}.
\end{equation}
where $c$ is the speed of light, $m_0$ the rest mass (independent of the velocity) of the particle, $e$ its charge, $E$ it's energy, $R$ the bending radius and $\epsilon_0$ the  vacuum permittivity.

Considering their low mass, the electrons are very good candidates to produce high energy radiation.

The most important properties of the synchrotron radiation are its brilliance and brightness. \todo{define...} The brightness is defined as [...] and the brilliance as [...]

The main purpose of BESSY II is to provide a light radiation with stable brilliance and brightness over time. To achieve this, the light source itself must be very stable as well.
Therefore a significant attention is drawn to the control of the storage ring.

2) Chain of acceleration + purpose
\section{Storage ring}
The purpose is to... not to collide or get highest energy => Stable => Storage ring

\todo[inline]{Purpose + Physics of the accelerator = 3-5 pages}
\section{Orbit and distortions}
\todo[inline]{Physics + examples = 1/2 - 1 pges}
