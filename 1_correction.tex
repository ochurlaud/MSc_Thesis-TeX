% !TeX spellcheck = en_US

\chapter{Orbit correction}
\label{sec:correction}
\section{Some documented methods}

Several corrections methods are well documented in the literature. The most common ones are the best corrector method, ...

\subsection{Most effective corrector method}
This method is based on the fact that orbit shifts are often caused by strong localized disturbances. Its goal is to correct particularly each disturbance.

\subsubsection{Principle}

Given a distorted orbit, the optimal gain for each corrector is calculated by a mean square error algorithm (see \eqref{eq:gain_bestcorr}, \cite{book:wille}). The corrector which provides the best correction is selected: it is the most effective corrector.

Let's assume that the $i$th corrector, at position $s_i$, has the optical parameters $\beta_i$, $\alpha_i$ and $\Psi_i$, and that $m$ monitors are set around the orbit with parameters $\beta_j$, $\alpha_j$ and $\Psi_j$ ($1 \leq j \leq m$).

The gain $\kappa_i$ of a this corrector is obtained minimizing the function
\todo{Explain from chap 1.3}
\begin{equation}
    \label{eq:gain_bestcorr}
    f_i(\kappa_i) = \sum\limits_{j=1}^{m} (u_j-x_{ij}(\kappa_i))^2 
                  = \sum\limits_{j=1}^{m} (u_j- \kappa_i h_{ij})^2
\end{equation}

with, if $\Delta \Psi_{ij} := \Psi_j-\Psi_i$,

\begin{align}
    x_{ij}(\kappa_i) &= \kappa_i h_{ij} \nonumber\\
                     &= \kappa_i \frac{\sqrt{\beta_i \beta_j}}{2}
                         \left[
                             \frac{\cos(\Delta \Psi_{ij}) - 2\alpha_i \sin(\Delta\Psi_{ij})}
                                  {\tan (\pi Q)} + \sin (\Delta\Psi_{ij})
                         \right]
\end{align}

\subsubsection{Iterative version}
When the most effective corrector is found, the process is reiterated on the corrected orbit with the remaining correctors. 

By doing this until all corrector are used (or that adding a correction does not improve the orbit) a comprehensive correction is reached.

\subsubsection{Practical issue}
The problem of this method is that each corrector must be tested once, and this for each iteration: the initialization of the correction is long and is then fixed.

\subsection{Fourier analysis}

\subsection{Matrix inversion}
Can be obtained empirically or using the theory. However, because accelerator diff from theoretical model, empirical is better. \cite{art:decker-1991}

Brute force
\todo[inline]{Best corrector, "...", inverse pb = 3 pages}
\section{State of the art at BESSY II}
\todo[inline]{2 pages}
\section{Acquisition of the transfer matrix}
