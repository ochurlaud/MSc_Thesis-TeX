% !TeX spellcheck = en_US

\chapter{Orbit correction}
\label{sec:correction}
\section{Some documented methods}

Several global corrections methods are well documented in the literature. The most common ones are the best corrector method and the response matrix method (see Sections~\ref{ssec:most_effective_corr}) and \ref{ssec:response_matrix}). 

Local orbit bumps (presented in Section~\ref{ssec:orbit_bump}) allow local correction and are used to change the path of the orbit (during the injection time for example).

\subsection{Local orbit bumps}
\label{ssec:orbit_bump}
\todo[inline]{see Wille p286}

\subsection{Most effective corrector method}
\label{ssec:most_effective_corr}
This method is based on the fact that orbit shifts are often caused by strong localized disturbances. Its goal is to correct particularly each disturbance.

\subsubsection{Principle}

Given a distorted orbit, the optimal gain for each corrector is calculated by a mean square error algorithm (see \eqref{eq:gain_bestcorr}, \cite{book:wille}). The corrector which provides the best correction is selected: it is the most effective corrector.

Let's assume that the $i$th corrector, at position $s_i$, has the optical parameters $\beta_i$, $\alpha_i$ and $\Psi_i$, and that $m$ monitors are set around the orbit with parameters $\beta_j$, $\alpha_j$ and $\Psi_j$, and read a displacement $u_j$ from the reference orbit ($1 \leq j \leq m$).

The strength $\kappa_i$ of the field at the position $s_i$ is obtained minimizing the function
\todo{Explain in chap 1.3}
\begin{equation}
    \label{eq:gain_bestcorr}
    f_i(\kappa_i) = \sum\limits_{j=1}^{m} (u_j-x_{ij}(\kappa_i))^2 
                  = \sum\limits_{j=1}^{m} (u_j- \kappa_i h_{ij})^2
\end{equation}

with, if $\Delta \Psi_{ij} := \Psi_j-\Psi_i$,

\begin{align}
    \label{eq:hij}
    x_{ij}(\kappa_i) &= \kappa_i h_{ij} \nonumber\\
                     &= \kappa_i \frac{\sqrt{\beta_i \beta_j}}{2}
                         \left[
                             \frac{\cos(\Delta \Psi_{ij}) - 2\alpha_i \sin(\Delta\Psi_{ij})}
                                  {\tan (\pi Q)} + \sin (\Delta\Psi_{ij})
                         \right].
\end{align}

It follows that 
\begin{equation}
    \kappa_i = \frac{\sum\limits_{j=1}^m u_j h_{ij}}{\sum\limits_{j=1}^m h_{ij}^2}
\end{equation}

The $i$th corrector is attributed the gain $-\kappa_i$ to compensate the field.

\subsubsection{Iterative version}
When the most effective corrector is found, the process is reiterated on the corrected orbit with the remaining correctors. 

By doing this until all corrector are used (or that adding a correction does not improve the orbit) a comprehensive correction is reached.

\subsubsection{Practical issue}
The problem of this method is that each corrector must be tested once, and this for each iteration: the initialization of the correction is long and is then fixed.

Moreover, the correction is less efficient than the other ones presented here~\cite{book:wille}.

\subsection{Response matrix}
\label{ssec:response_matrix}

This section is mainly based on~\cite{book:wille}, \cite{art:decker-1991} and \cite{art:plouviez-1999}.

The response matrix $\mat{S}$ is defined by the equation $\vec{X} = \mat{S} \vec{\Theta}$ where $\vec{\Theta}$ is the \emph{kick vector} and $\vec{X}$ the vector of \emph{orbit change}. If the accelerator has $M$ monitors and $N$ correctors, $\mat{S}$ is a $M \times N$ matrix.

Every coefficient $\mat{S}_{ij}$ is the orbit change at the position $s_i$ (of the $i$th monitor), for a kick of unity 1 at the position $s_j$ (of the $j$th corrector).

This matrix can be theoretically calculated by using the accelerator model and physics: $\mat{S}_{ij} = h_{ij}$ with the formula in \eqref{eq:hij}. However, the actual storage ring is never exactly the same as the designed one, because of several inaccuracies in its building. A more common way of constructing the response matrix is empirical: each measure cycle, a corrector is given a unitary value and the monitor are read, providing a column of the response matrix.

To do the correction, the inverse problem must be solved. We know the orbit change that we need to obtain a cleaner orbit, using the inverse of the response matrix will provide the correction to apply. Since it's very common to have more monitors than correctors, the matrix is not square. A \emph{singular value decomposition} (SVD) is used on the matrix to provide a pseudo-inversion.

Using the SVD also allows to use only the most significant singular values in the correction.\todo{why do we want this?}

\section{State of the art at BESSY II}
\todo[inline]{2 pages}
\section{Acquisition of the transfer matrix}
