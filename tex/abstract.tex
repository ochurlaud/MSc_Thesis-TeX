% !TeX encoding = UTF-8
% !TeX spellcheck = de_DE
% !TeX root = ../MasterThesis_OlivierChurlaud_2016.tex

\begin{abstract}
    BESSY~II is a \SI{240}{\meter} electron storage ring in Berlin, Germany, aimed at producing stable and high-energy light rays by synchrotron radiation. It is a light source which is used by various scientific projects and must therefore provide a very stable and reliable light. This means that the electrons must be well focused in the center of their vacuum-chamber. One of the key requirements is thus to be able to control and remove perturbations from the electron beam.
    
    The electron beam is described as an orbit, characterized by these parameters: the beta function $\beta$ which defines the size of the beam, the tune $Q$ which is the number of oscillations per revolution done by the beam around its ideal path and the phase $\Psi$ of the orbit, which represents the magnet's influence on the orbit.
    
    From these parameters, it is possible to derive several correction methods, based on the activation of corrector magnets around the orbit. In this work, the inverse problem was solved thanks to the pseudo-inversion of the response matrix representing the changes of the orbit when a corrector magnet is activated. This, used with a PID correction, already provided an acceptable stable orbit. Additionally, to correct some specific harmonic perturbation, a finite impulse response (FIR) filter that generates the opposite harmonic was implemented, removing most of the targeted perturbation. A simulation environment was also designed to help improve the current correction.
    
    Finally, algorithms were implemented to localize harmonic and non-harmonic static sources of the perturbations, at a precision of 2 to \SI{4}{\meter} in the case of a vertical perturbation, and \SI{10}{\meter} in the horizontal case. This makes it possible to manually isolate or remove the source, instead of correcting its effects.
\end{abstract}
\cleardoublepage

\renewcommand{\abstractname}{Zusammenfassung}
\begin{abstract}
    BESSY~II ist ein Elektronenspeicherring  mit \SI{240}{\meter} Umfang in Berlin (DE), der sehr stabile und hoch energetische Lichtstrahlen durch Synchrotronstrahlung produziert. Dieses Licht wird von mehreren verschiedenen Forschungsprojekten benutzt und muss deshalb hochstabil und betriebssicher sein. Das bedeutet, dass die Elektronen auf einer Bahn bleiben müssen. Eine der Hauptanforderungen ist deshalb, dass etwaige Störungen des Elektronenstrahls sowohl kontrollierbar, als auch behebbar sein müssen.
    
    Der Elektronenstrahl wird durch einen Orbit beschrieben, der durch die folgende Parameters gekennzeichnet ist: die Betafunktion $\beta$ definiert die Strahlengröße, der Tune $Q$ ist die Anzahl von Schwingungen, die der Strahl entlang seines idealen Wegs in einem Umlauf durchläuft und die Phase $\Psi$ stellt den Einfluss der Magnete auf den Orbit dar.
    
    Verschiedene Korrekturmethoden können durch die jeweilige Aktivierung von Korrektormagneten entlang der Umlaufbahn verwendet werden. In dieser Arbeit wurde  das inverse Problem durch die Pseudo-Inversion derjenigen	Reaktionsmatrix gelöst, die	Veränderungen der	Umlaufbahn beschreibt,	wenn	Korrektormagneten	aktiviert	werden. Die Implementierung dieser Technik inklusive eines PID Regler hat bereitseine akzeptable stabile Umlaufbahn garantiert. Zusätzlich wurden spezifische harmonische Störungen durch einen finite impulse response Filter (FIR) beseitigt.
    
    Um die aktuelle Korrektur zu verbessern, wurde den Speicherring modelliert und eine Simulation entworfen.
    
    Anschließend wurden Algorithmen implementiert, um harmonische und nicht harmonische statische Störungsquellen zu lokalisiert. Die Genauigkeit der Lokalisierung erreicht im Falle einer vertikalen  circa \SI{4}{\meter} und im Falle einer horizontalen Störung \SI{10}{\meter}. Dadurch können diese Störungsquellen manuell isoliert oder entfernt werden, anstatt nur ihre Wirkungen zu korrigieren.
\end{abstract}
