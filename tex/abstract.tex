% !TeX encoding = UTF-8
% !TeX spellcheck = en_US
% !TeX root = ../MasterThesis_OlivierChurlaud_2016.tex

\begin{abstract}
BESSY~II is a \SI{240}{\meter} circular electron particle accelerator in Berlin, DE, aimed at producing stable and high-energy light ray by synchrotron radiation. This light source is used by various scientific projects and must therefore provide a very stable and reliable light, which means that the electrons must be very focused in middle of their vacuum-chamber. One of the key requirement is thus to be able to control and remove perturbations from the electron beam.

The electron beam is described as an orbit, characterized by some parameters: the beta function $\beta$ which defines the size of the beam, the tune $Q$ which is the number of oscillations done by the beam around its ideal path in one revolution and the phase $\Psi$ of the orbit, which represents the magnet influence on the orbit.

From this can be derived several correction methods, based on the activation of corrector magnets around the orbit. An inverse problem can being solved thanks to the pseudo-inversion of the response matrix representing the changes of the orbit when a corrector magnet is activated. This, used with a PID correction, already provides an acceptable stable orbit. Additionally, to correct some specific harmonic perturbation, a finite impulse response (FIR) filter that generates the opposite harmonic is implemented, removing most of the targeted perturbation. A simulation environment is designed to help improving the current correction.

Finally, algorithms are written to localize harmonic and non-harmonic static sources of the perturbations, at a precision of 2 to \SI{4}{\meter} in the case of a vertical perturbation, and \SI{10}{\meter} in the horizontal case. This allows to manually isolate or remove the source, instead of correcting its effects.
\end{abstract}

\renewcommand{\abstractname}{Zusammenfassung}
\begin{abstract}
    \todo[inline]{TODO: abstract Allemand}
\end{abstract}
