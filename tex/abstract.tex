% !TeX encoding = UTF-8
% !TeX spellcheck = de_DE en_US
% !TeX root = ../MasterThesis_OlivierChurlaud_2016.tex

\begin{abstract}
    BESSY~II is a \SI{240}{\meter} electron storage ring in Berlin, DE, aimed at producing stable and high-energy light ray by synchrotron radiation. This light source is used by various scientific projects and must therefore provide a very stable and reliable light, which means that the electrons must be very focused in middle of their vacuum-chamber. One of the key requirement is thus to be able to control and remove perturbations from the electron beam.
    
    The electron beam is described as an orbit, characterized by some parameters: the beta function $\beta$ which defines the size of the beam, the tune $Q$ which is the number of oscillations done by the beam around its ideal path in one revolution and the phase $\Psi$ of the orbit, which represents the magnet influence on the orbit.
    
    From this can be derived several correction methods, based on the activation of corrector magnets around the orbit. In this work, the inverse problem was solved thanks to the pseudo-inversion of the response matrix representing the changes of the orbit when a corrector magnet is activated. This, used with a PID correction, already provided an acceptable stable orbit. Additionally, to correct some specific harmonic perturbation, a finite impulse response (FIR) filter that generates the opposite harmonic was implemented, removing most of the targeted perturbation. A simulation environment was also designed to help improving the current correction.
    
    Finally, algorithms were implemented to localize harmonic and non-harmonic static sources of the perturbations, at a precision of 2 to \SI{4}{\meter} in the case of a vertical perturbation, and \SI{10}{\meter} in the horizontal case. This allows to manually isolate or remove the source, instead of correcting its effects.
\end{abstract}
\cleardoublepage

\renewcommand{\abstractname}{Zusammenfassung}
\begin{abstract}
    BESSY~II ist ein Elektronenspeicherring in Berlin (DE), der mit einem Umfang von \SI{240}{\meter} sehr stabile und hoch energetische Lichtstrahlen durch Synchrotronstrahlung produziert. Diese Lichtquelle wird von mehreren verschiedenen Forschungsprojekten benutzt und muss deshalb hochstabiles und betriebssicheres Licht liefern. Das bedeutet, dass die Elektronen im Zentrum des Vakuumzimmers fokussiert bleiben müssen. Eine der Hauptanforderungen ist deshalb, dass etwaige Störungen des Elektronenstrahls sowohl kontrollierbar, als auch behebbar müssen.
    
    Der Elektronenstrahl wird als ein Orbit beschrieben, der durch die folgende Parameters gekennzeichnet ist: die Betafunktion $\beta$ definiert die Strahlengröße, das Tune $Q$ ist die Anzahl von Schwingungen, dass der Strahl entlang seinem idealen Weg in einem Umlauf durchläuft und die Phase stellt den Einfluss der Magneten auf den Orbit dar.
    
    Verschiedene Korrekturarten können durch die jeweilige Aktivierung von Korrektormagneten entlang der Umlaufbahn abgeleitet werden. In diese Arbeit wurde  das inverse Problem durch die Pseudo-Inversion derjenigen	Reaktionsmatrix gelöst, die	Veränderungen der	Umlaufbahn	darstellt,	wenn	Korrektormagneten	aktiviert	werden. Die Implementation dieser Technik inklusive einer PID Korrektor hat schon eine zumutbar stabile Umlaufbahn garantiert. Zusätzlich wurden spezifische harmonische Störungen durch einen Filter mit endlicher Impulsantwort (FIR) beseitigt, der die gegenteiligen Harmonischen generiert. Eine simulierte Umgebung wurde auch entworfen, um die aktuelle Korrektur zu verbessern.
    
    Zum Schluss wurden Algorithmen implementiert,um harmonische und nicht harmonische statische Störungsquellen zu lokalisiert. Die Präzision der Lokalisierung erreicht circa \SI{4}{\meter} im Falle einer vertikalen und \SI{10}{\meter} im Falle einer horizontalen Störung. Dadurch können diese Störungsquellen manuell isoliert oder entfernt werden, anstatt nur ihre Wirkungen zu korrigieren.
\end{abstract}
