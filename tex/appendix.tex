% !TeX encoding = UTF-8
% !TeX spellcheck = en_US
% !TeX root = ../main.tex

\section{Brightness and brilliance}
\label{apx:brightness_brilliance}
In the following section, the $\theta$ subscript is either $x$ or $y$.

The quality of the beam is described by the \emph{brightness} and the \emph{brilliance}.

Let F be the \emph{flux} of photons, normalized to a beam current of 1~A
\begin{equation}
F = \frac{\text{photons}}{\text{s 0.1\% BW A}},
\end{equation}

The brightness describes the angular divergence of the beam (given by $\sigma_\theta'=\sqrt{\frac{\varepsilon_\theta}{\beta_\theta}}$). It is defined as:
\begin{equation}
S = \frac{F}{2 \pi \sigma'_x \sigma'_y} = \frac{F \sqrt{\beta_x \beta_y}}{2 \pi \sqrt{\varepsilon_x \varepsilon_y}} = \frac{\text{photons}}{\text{s 0.1\% BW mrad$^2$ A}}
\end{equation}
while the brilliance includes also the transverse dimensions ($\sigma_\theta=\sqrt{\varepsilon_\theta \beta_\theta}$):
\begin{equation}
B = \frac{F}{4 \pi^2 \sigma_x \sigma_y \sigma'_x \sigma'_y} = \frac{F}{4 \pi^2 \varepsilon_x \varepsilon_y} = \frac{\text{photons}}{\text{s 0.1\% mm$^2$ BW mrad$^2$ A}}
\end{equation}

These definitions vary in literature. These are taken from the book of K.~Wille~\cite{book:wille}, and apply to Gaussian-shaped electron beams. The invariant idea is that both values are determinated by the beam emittance $\sigma$: the design of the accelerators and the correction are aimed at obtaining the smallest emittance $\varepsilon_\theta$ as possible.

\section{Principal components analysis -- Karhunen-Loève transform (KLT)}
\label{apx:KLT}

Let $\mat{X}$ be a signal with n dimensions.

\begin{equation}
\mat{X} = \left[
\begin{array}{@{}c|c|c|c@{}}
&&&\\ \vec{X}_1 & \vec{X}_2 & \dots & \vec{X}_n \\ &&&
\end{array}
\right]
\end{equation} 

The covariance matrix is extracted
\begin{equation}
\mat{A} = \text{covar}(X) 
\end{equation}
and diagonalized so that the eigenvalues are sorted in a decreasing order (the first one being the most significant). The covariance being symmetric, it is always diagonalizable in $\mathbb{R}$.
\begin{equation}
\mat{D} = \mat{P}^{t} \mat{A} \, \mat{P} 
\end{equation}

The principal components are 
\begin{equation}
\hat{\mat{X}} = \mat{P}^{t} \mat{X}
\end{equation}
and in the first column of the matrix is the most significant one.
